% Papier à en-tête LORIA
% Version du 4 avril 2017
% Infos et bugs : gerald.oster@loria.fr

\documentclass[a4paper,11pt]{letter}

% Paquets généraux
\usepackage[utf8]{inputenc}     % Codage du fichier TeX
\usepackage[T1]{fontenc}        % Codage du fichier PDF (les paquets cm-super ou lmodern doivent être installés pou obtenir un bon résultat)
\usepackage{graphicx}           % Logos et autres illustrations
\usepackage{fancyhdr}           % En-têtes et pieds de page
\usepackage{color}              % Couleur
\usepackage{ifthen}             % Condition
\usepackage[percent]{overpic}
\usepackage{helvet}             % Police Helvetica
\usepackage[french]{babel}      % Courrier en français
% \usepackage[english]{babel}   % Courrier en anglais
% Papier à en-tête LORIA
% Version du 4 avril 2017
% Infos et bugs : gerald.oster@loria.fr

% La définition de ce style est une adaptation du style réalisé par Vincent Mazet <vincent.mazet@unistra.fr>
% pour le laboratoire ICube.

% Mise en page
\voffset -1in
\topmargin 10mm
\headheight 23mm %5mm
\headsep 10mm
\textheight 205mm %210mm
\footskip 38mm
\hoffset -1in
\oddsidemargin 50mm
\evensidemargin 50mm
\textwidth 140mm
\marginparsep -180mm
\marginparwidth 35mm
\parindent 0mm

% En tête et pieds de page
\pagestyle{fancy}
\renewcommand{\headrulewidth}{0pt}

% Police sans-serif
\renewcommand{\familydefault}{\sfdefault}

% Couleurs
\definecolor{bleu}{RGB}{0,94,168}
\definecolor{gris}{RGB}{88,88,90}
\definecolor{loria-gray}{RGB}{166,166,166}

% Logos
\newcommand{\loriasign}{\includegraphics[height=30mm]{loria-style/loria-sign.pdf}\vspace*{-10mm}}
\newcommand{\loria}{\includegraphics[height=30mm]{loria-style/loria.pdf}\vspace*{-10mm}}

% Marge (expéditeur, suivi, adresse)
\newcommand{\marge}{%
  \vspace*{-235mm}\hspace*{-40mm}\vbox{
    \vtop to  45mm {\vfil\usebox{\exped}\vfil}
    \vbox to  30mm {\vfil}
    \vbox to  30mm {\vfil\usebox{\suiv}\vfil}
    \vbox to 100mm {\vfil}
%    \vbox to  20mm {\vfil\usebox{\adress}}
  }
}

% Tutelles
\newcommand{\tutelles}{
  \begin{tabular}{@{}r@{}}%
  \includegraphics[height=1cm]{loria-style/cnrs.pdf}\hskip5mm\relax%
  \includegraphics[height=1cm]{loria-style/ul.pdf}\hskip5mm\relax%
  \iflanguage{french}{%
    \includegraphics[trim={7mm 6mm 7mm 6mm}, clip, height=1cm]{loria-style/inria-scientifique-fr.pdf}%
  }{%
    \includegraphics[trim={7mm 6mm 7mm 6mm}, clip, height=1cm]{loria-style/inria-scientifique-uk.pdf}%
  }
  \end{tabular}%
}

\newcommand{\addressfooter}{\parbox[c]{8.5cm}{%                             % Adresse laboratoire
  \sf \footnotesize \color{loria-gray}
  Campus scientifique\\
  BP 239 -- 54506 Vandœuvre-lès-Nancy Cedex\\
  Tél : 03 83 59 20 00 --- \href{https://www.loria.fr/}{\color{loria-gray}www.loria.fr} \\
  \raisebox{-.2em}{\includegraphics[height=1em]{loria-style/twitter.pdf}}~\href{https://twitter.com/Loria_nancy}{\color{loria-gray}Loria\_nancy}
  }
}

% Éléments de la lettre
\lhead{\hspace*{-40mm}\ifthenelse{\value{page}=1}{\loriasign}{\loria}}      % Logo LORIA
\chead{}                                                                    %
\lfoot{\ifthenelse{\value{page}=1}{\marge}{}}                               % Expéditeur, suivi, adresse
\cfoot{\ifthenelse{\value{page}=1}{%                                        % Tutelles
  \hspace*{-50mm}%
  \begin{overpic}[width=20.5cm]{loria-style/loria-wave}
   \put (14,6) {\addressfooter}
   \put (52,14) {\tutelles}
  \end{overpic}
  }{%
  \hspace*{-50mm}%
  \begin{overpic}[width=20.5cm]{loria-style/loria-wave}
  \end{overpic}
  }%
}
\rfoot{\ifthenelse{\value{page}=1}{}{\raisebox{15mm}{\thepage}}}            % Numéro de page

% Expéditeur
\newsavebox{\exped}
\newcommand{\expediteur}[3]{\savebox{\exped}{\parbox{50mm}{%
  \textcolor{bleu}{#1 \textbf{#2}} \\[1ex]\scriptsize#3}}}

% Suivi de l'affaire
\newsavebox{\suiv}
\newcommand{\suivi}[1]{\savebox{\suiv}{\parbox{35mm}{\scriptsize%
  \textcolor{bleu}{Affaire suivie par~:} \\[.5ex]#1}}}

% Adresse du laboratoire
%\newsavebox{\adress}
%\newcommand{\adresse}[1]{\savebox{\adress}{\parbox{75mm}{%
%  \scriptsize\color{gris}\textbf{%
%  LORIA --- UMR 7503\\
%  Campus Scientifique\\BP 239 ---  54506 Vandœuvre-lès-Nancy Cedex --- France}\\
%  #1}}
%  }

% Lieu, date
\newcommand{\lieudate}[1]{\rhead{\ifthenelse{\value{page}=1}{#1}{}}}

% Destinataire
\newcommand{\destinataire}[1]{\vspace*{-20mm}\vbox to 4cm {\vfil \hfill\hbox to 100mm{\parbox{90mm}{#1}}\vfil}\vspace*{20mm}}
 % Style de la lettre à en-tête
% \usepackage{fontspec}         % A décommenter si compilation avec XeLaTeX
\usepackage{url}

\pdfminorversion=7

\usepackage[
  pdftitle={Courrier},
  pdfauthor={Gerald Oster},
%  pdfsubject={},
%  pdfkeywords={},
  colorlinks=true,
  pdfstartview=Fit,
  bookmarks=false,
  hyperindex,
  linkcolor=black,
  citecolor=black,
  urlcolor=bleu
 ]{hyperref}

\begin{document}

% Lieu et date
\lieudate{Vandœuvre-lès-Nancy, le \today}

\newcommand{\email}[1]{\href{mailto:#1}{\nolinkurl{#1}}}

% Expéditeur
\expediteur{Gérald}{Oster}{%
  Maître de conférences, \\
  Tél~: +33 (0)3 83 59 30 77 \\
  \email{gerald.oster@loria.fr}
}

% Affaire suivie par...
\suivi{%
  Prénom Nom \\
  Tél~: \\
  Mél~: \\
  Réf.
}

% Destinataire
\destinataire{%
  LORIA \\
  Bâtiment B, équipe COAST \\
  Campus Scientifique, BP 239 \\
  54506 Vandœuvre-lès-Nancy Cedex --- France
}

% Objet
Objet~: rédaction d'une lettre avec papier à en-tête en \LaTeX{}

\bigskip\bigskip

Madame, Monsieur,

\bigskip\bigskip

La version \LaTeX{} du papier à en-tête du laboratoire LORIA est configurable avec les éléments suivants~:
\begin{itemize}
  
  \item \texttt{$\backslash{}$lieudate\{\dots\}} permet d'indiquer
  le lieu et la date de rédaction en haut à droite sur la première page~;

  \item \texttt{$\backslash{}$expediteur\{\textit{Prénom}\}\{\textit{Nom}\}\{\dots\}}
  s'écrit dans la marge et identifie l'auteur de la lettre. Indiquez vos prénom, nom et coordonnées~;

  \item \texttt{$\backslash{}$suivi\{\dots\}} permet de donner des informations sur la personne suivant le dossier~;

  \item \texttt{$\backslash{}$adresse\{\dots\}} contient l'adresse du laboratoire,
  à personnaliser suivant le site où on se trouve~;

  \item \texttt{$\backslash{}$destinataire\{\dots\}} correspond au destinataire du courrier~;

  \item le LORIA possède trois tutelles. Le CNRS et l'Université de Lorraine et Inria doivent apparaître.

\end{itemize}

\bigskip

Ce document nécessite les logos des tutelles (fichiers \texttt{cnrs.pdf}, \texttt{ul.pdf}, \texttt{loria.pdf}, \texttt{loria-sign.pdf}, \texttt{loria-wave.pdf},  \texttt{inria-scientifique-*.pdf} et \texttt{twitter.pdf}) et les paquets suivants~:
\begin{itemize}
  
  \item fontenc, pour créer des fichiers PDF avec des polices correctes
  (les paquets cm-super ou lmodern doivent par ailleurs être installés pour obtenir un bon résultat)~;

  \item graphicx pour afficher des images (en l'occurrence, les logos)~;
  
  \item overpic pour afficher la vague de bas de pages~;

  \item fancyhdr, pour pouvoir modifier les en-têtes et pieds de page~;

  \item color, pour prendre en charge les couleurs de la charte graphique~;

  \item ifthen, pour gérer des conditions~;

  \item helvet pour écrire avec la police Helvetica~;

  \item babel, avec l'option french ou english pour prendre en comptes les particularités de la langue.

\end{itemize}
Le fichier loria-style.tex est indispensable~: il contient les commandes nécessaires à la compilation.

\bigskip

Les logos inclus sont au format PDF. De ce fait, la compilation est à faire avec les commandes \texttt{pdflatex} ou \texttt{xelatex}.

\bigskip

Ce document correspond à la version du \today.
Pour toute information supplémentaire, ou pour faire remonter une correction, veuillez m'envoyer un courriel à \email{gerald.oster@loria.fr}. Ce style est une adaptation du style réalisé par Vincent Mazet pour le laboratoire iCube.

\bigskip

Veuillez agréer, Madame, Monsieur, mes salutations distinguées.

\bigskip\bigskip

\hfill Gérald Oster

\end{document}
